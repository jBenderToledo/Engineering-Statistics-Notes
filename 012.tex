
\documentclass[]{article}
\usepackage{amsmath}

%opening
\title{Engineering Statistics Lecture XII}
\author{Jonathan Bender}
\date{October 15, 2019}

\begin{document}
	
	\maketitle
	
	\begin{abstract}	
		HW \#2 is due October 15, 2019:
		\begin{itemize}
			\item Section 2.3 \#23-37 odd
			\item Section 2.4 \#49-65 odd
			\item Section 2.5 \#73-93 odd
		\end{itemize}
	
		NO CLASS THURSDAY, OCTOBER 10, 2019
	\end{abstract}

	\section{Mathematical Expectiation}
	Written as E[argument], expected values give us an idea of what to "expect" of the argument involving a random variable. It is not any of the modes, usually. It is the average value given a PDF.
	
	Suppose E[g(x)] is the expectation of g(X) for some random variable X with a PDF f(x):
	$$E[g(x)]=\sum_{all\ x}g(x)f(x)=\int_{all\ x}g(x)f(x)dx$$
	with use of the discrete or continuous sum depending upon the set upon which X operates.
		\subsection{Mode}
		Local maxima in PDFs are called "modes"
	
\end{document}

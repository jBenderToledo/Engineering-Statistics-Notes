
\documentclass[]{article}
\usepackage{amsmath}
\usepackage{amssymb}

%opening
\title{Engineering Statistics Lectures XXI}
\author{Notes by Jonathan Bender}
\date{December 5, 2019}

\begin{document}
	
	\maketitle
	
	\begin{abstract}
		Final opportunity given December 5, 2019 -- **due December 9, 2019 at 6:00 PM**.
	\end{abstract}

	\section{Review!}
		In hypergeometric, if n $< \dfrac{N}{20}$, the binomial approximates the hypergeometric.
		
	\section{Stuff needed for Question \#10!!!!!}
		Normal approximation to binomial. Many times, we've only said that the binomial is only good for what's in the table and constant values of p. What if you use a table and you have a value of p that can't be done with a table? What if we had a binomial with the same value? What would it look like?
		
		Well, the normal curve will approximate the binomial; the approximation is best for large N (a lot of trials) and $p \approx 0.5$;
		
		For a binomial b(x;n,p), use f(x;$\mu,p$) given $\mu = np, \sigma = \sqrt{npq}$. To refresh, n is the number of trials taken and p is the chance of success in an individual trial.
\end{document}

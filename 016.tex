
\documentclass[]{article}
\usepackage{amsmath}

%opening
\title{Engineering Statistics Lectures XVI}
\author{Notes by Jonathan Bender}
\date{November 7, 2019; November 12, 2019}

\begin{document}
	
	\maketitle
	
	\begin{abstract}
		HW \#3 due today.
		Opportunity \#1 given November 21, 2019 and due November 26, 2019.
		Final opportunity given sometime during the week of December 10, 2019.
		
		Opportunity \#0 fixes are due Thursday, November 14.
	\end{abstract}

	Chapter 5 begins today!
	
	\section{Discrete Probability Functions}
		\subsection{Uniform}
			Each of the k values of x has equal probability. In example, rolling a die, selecting a 3-digit integer at random, et cetera.
		\subsection{Mean}
			For a set of k items indexed within X,
			$$\mu_x = \sum_{x_i \in X}\dfrac{x_i}{k}$$
		\subsection{Sample variance / population variance}
			For this same definition as above,
			$$\sigma_x^2 = \dfrac{1}{k-1}\sum_{x_i \in X}(x_i - \mu_x)^2$$
	
	\pagebreak
	\section{Binomial PDF}
		Four conditions for a binomial PDF exist:
		
		\begin{enumerate}
			\item Some integer n fixed trials.
			\item Each trial is a success or failure (S or F).
			\item The trials are independent -- the occurrences of one trial do not affect the state of any subsequent trials.
			\item P(S) = p = constant; the probability of success is in and of itself constant.
		\end{enumerate}
	
		Very short version --  A binomial PDF is defined around constant, unchanging conditions by which one trial is completely independent of another. Any subject of change means that the circumstances are NOT a binomial PDF! A random variable X is a binomial random variable if the four conditions are met AND X is the number of successes in some number of n trials.
		
	\section{Example: Unfair coin.}
		Flip an unfair coin 5 times such that S = tails and P(S) = $\frac{3}{4}$. What's P(X=4)?
		
		Define p to be P(S) and q to be P(F) = 1 - P(S).
		
		\begin{align*}
			P(X=4) &= \binom{5}{4} * (\dfrac{3}{4})^4 * (1 - \frac{3}{4}) \\
			       &= 5 * \frac{81}{256} * \frac{1}{4} \\
			       &= \frac{405}{1024} \\
			P(X=4) &\approx 0.3955 \\\\
			P(X<3) &= \sum_{x=0}^{2} \binom{5}{x}p^x q^{(5-x)} \\
			       &= \sum_{x=0}^{2} \frac{5!}{x!(5-x)!} (\frac{3}{4})^x(\frac{1}{4})^{(5-x)} \\
			       &= \frac{53}{512}
		\end{align*}
		
		If for any reason you are given a CDF table and you want to find an individual probability, take the value from the CDF table that you want and subtract the previous instance of the CDF.
		
		\pagebreak
		
		What if we make it 10 times instead, and P(S) = 0.6?
		
		\subsection{A. What is $\mu$?}
			$\mu = np = 10 * 0.6 = 6$.
			6 times on average -- we expect 6 successes and 4 failures on average.
		\subsection{B. What is $\sigma_x^2$?}
			$\sigma^2 = np(1-p) = npq = 10 * 0.6 * 0.4 = 2.4$.
			So, the population variance is around 2.4.
		\subsection{C. What is P($X \leq 3$)?}
			\begin{align*}
				 &\sum_{x=0}^{3}\binom{10}{x}(0.6)^x(0.4)^{10-x} \\
				=& 0.05476
			\end{align*}
		\subsection{D. What is P($X > 5$)?}
			= 1 - P($X \leq 5$).
			= 1 - 0.3669
		\subsection{E. What is P($X \in [3, 7]$)?}
			P(X=7) - P(X=3) $\approx$ 0.7779
\end{document}

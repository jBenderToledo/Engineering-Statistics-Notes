
\documentclass[]{article}
\usepackage{amsmath}

%opening
\title{Engineering Statistics Lectures XV and XVI}
\author{Notes by Jonathan Bender}
\date{October 24, 2019; October 29, 2019}

\begin{document}
	
	\maketitle
	
	\begin{abstract}
		Bell Labs -- Subsidiary of AT\&T, company invented many things ranging from the BJT to the general transistor to the telephone (Namesake -- Alexander Graham Bell) to the IC. Basically died at the late 70's AT\&T split.
	\end{abstract}
	
	\section{Reprisal of Joint Probability}
		For three variables, $$P(X=x, Y=y) = \dfrac{J_x(x)J_y(y)J_z(x,y)}{\#\ of\ possible\ cases}$$
		
		Where $J_x(x)$ refers to a joint probability element which concerns x. and $J_z(x,y)$ refers to a joint probability element which concerns z, a dependent variable upon x and y.
		Generally, these are $_nC_r$ functions such that the sum of all n and the sum of all k in the numerator match that of the denominator.
	
	\section{Marginal probabilities}
		What if we want the chance that, regardless of y, x is fixed?
		Well, good! We have a way of dealing with it!
		
		$$P(x) = \sum_{j \in Y}P(x,j)$$
		
		Where x is some fixed value and j iterates over the set of possible values for Y.
		For three independent variables,
		
		$$P(x) = \sum_{j \in Y}\sum_{k \in Z}P(x,j,k)$$
		
		Where, as before, j and k iterate over Y and Z. Think of P(x) to be a lesser-dimensional slice of the relevant space (plane corresponds to line, space corresponds to plane, hyperspace corresponds to space, etc.).
	
	\section{Means in multidimensional sets}
		$$\mu_X = \frac{P(x)}{\|X\|} = \dfrac{\sum_{j \in Y}P(x,j)}{\#\ of\ terms\ in\ X} = avg(P(x))$$
		
		Where P(x) is the marginal probability of x occurring.
	
	\pagebreak
		
	\section{Marathon I guess?}
		Spoz that the fraction of guys that complete a marathon is X, and the fraction of women is Y. Historically-speaking, $f(X=x, Y=y) = Axy\ s.t.\ 0\leq y\leq x\leq 1$.
		
		To get the probability between two values of x, we take the integral of the PDF f(x)  between those two values.
		
		To get the probability within two ranges of x and y respectively, we take the integral of a PDF f(x,y) within those ranges; the area under the surface.
		
		\subsection{A. find A.}
		
		\begin{align*}
			\int\int Axy\ dxdy &= 1\ \forall (x,y)\ s.t.\ 0\leq y\leq x\leq 1\\
			\to \int_{y=0}^{x=1} \int_{x=y}^{x=1} Axy\ dxdy &= 1 \\
			\to \int_{x=0}^{x=1} \int_{y=0}^{y=x} Axy\ dydx &= 1 \\
			\to \int_{x=0}^{x=1} \dfrac{A}{2}xy^2 |_{y=0}^{y=x}\ dy &= 1 \\
			\to \int_{x=0}^{x=1} \dfrac{A}{2}x^3 dx &= 1 \\
			\to \dfrac{A}{8}x^4 |_{x=0}^{x=1} &= 1 \\
			\to \dfrac{A}{8} &= 1 \\
			\to A &= 8 \\\\
			\to f(x,y) &= 8xy
		\end{align*}
		
		\pagebreak
		\subsection{B. Find g(x), h(y)}
		Find g(x):
		\begin{align*}
			g(x) &= \int_{y=0}^{y=x}f(x,y)dy \\
			\to g(x) &= 8\int_{y=0}^{y=x}xy\ dy \\
			         &= 4xy^2|_{y=0}^{y=x} \\\\
			\to g(x) &= 4x^3
		\end{align*}
		Find h(y):
		\begin{align*}
			    h(y) &= \int_{x=y}^{x=1}[f(x,y)]dx \\
			\to h(y) &= \int_{x=y}^{x=1}[8xy]dx \\
			         &= 4x^2y|_{x=y}^{x=1} \\
			         &= 4y(1 - y^2) \\\\
			\to h(y) &= 4y(1+y)(1-y)
		\end{align*}
		
		\pagebreak
		\subsection{C. Find $\mu_X,\ \mu_Y$}
		\begin{align*}
				\mu_X &= \int_{x=0}^{x=1}xg(x)dx \\
				      &= \int_{x=0}^{x=1}4x^4dx \\
				      &= \frac{4}{5}x^5|_{0}^{1} \\\\
			\to \mu_X &= \frac{4}{5}
		\end{align*}
		
		Find $\mu_Y$:
		
		\begin{align*}
				\mu_Y &= \int_{y=0}^{y=1}yh(y)dy \\
				      &= \int_{0}^{1} y * 4y(y-y^3)dy \\
				      &= \int_{0}^{1} 4(y^3 - y^5)dy \\
				      &= [y^4 - \frac{2}{3}y^6]_{y=0}^{y=1} \\\\
			\to \mu_Y &= \frac{1}{3}
		\end{align*}
		
		\pagebreak
		\subsection{D. What's P($0.1\leq X\leq 0.3,\ 0.2\leq Y\leq 0.4$)}
		
		Find the probability by summing up the contents under the curve. However, Y is strictly bounded by X: $Y\leq X$!! However, the same rule follows: $Y\not\geq X$. So, the bound becomes: $$0.2\leq X\leq 0.3, 0.2\leq Y\leq 0.3$$
		
		So, we take the integral as:
		\begin{align*}
			P(etc) &= \int_{x=0.2}^{x=0.3}\int_{y=0.2}^{y=0.3}f(x)dydx \\
			       &= \int_{x=0.2}^{x=0.3}\int_{y=0.2}^{y=0.3}8xydydx \\
			       &= \int_{x=0.2}^{x=0.3}[4xy^2]_{y=0.2}^{y=0.3}dx \\
			       &= \int_{x=0.2}^{x=0.3}4(\frac{9}{100}-\frac{4}{100})xdx \\
			       &= \frac{1}{5}*\frac{1}{2}[x^2]_{x=0.2}^{x=0.3} \\
			       &= \frac{1}{10}[\frac{9}{100} - \frac{4}{100}]
		\end{align*}
		
		\pagebreak
		\section{Brian has two pumps?}
		
		Spoz that Brian runs a gas station with two pumps.
		
		Let X be the fraction of time that a customer has to wait in line (only has two pumps) for the first pump. Let Y be the fraction of time that a customer has to wait in line for the other pump. Historically-speaking, after much data compiled and taken using python, the PDF f(x,y), given that 10 minutes is the maximum wait time possible:
		\begin{align*}
			f(x,y) = &\{A(2x + 3y)  &iff\   0\leq x\leq 1\}& \\
			         &\{            &       0\leq y\leq 1\}& \\
			         &\{            &                    \}& \\
			         &\{0           &       elsewhere    \}&
		\end{align*}
		
		\subsection{A. Find A}
			\begin{align*}
				A      &= \frac{1}{F(x,y)} \\
			\to F(x,y) &= \int_{y=0}^{y=1}\int_{x=0}^{x=1}(2x+3y)dxdy \\
			           &= \int_{y=0}^{y=1}(x^2 + 3xy)|_{x=0}^{x=1}dy \\
			           &= \int_{y=0}^{y=1}(1 + 3y)dy \\
			           &= [y + \frac{3}{2}y^2]_{y=0}^{y=1} \\
			           &= \frac{5}{2} \\
			           \\
			\to A      &= \frac{2}{5}
			\end{align*}
		
		\pagebreak
		\subsection{B. Find $\mu_X, \mu_Y$}
		Find $\mu_X$ by finding g(x):
		
		\begin{align*}
			g(x)  &= \int_{y=0}^{y=1}\frac{2}{5}(2x+3y)dy \\
			      &= \frac{2}{5}[2xy + \frac{3}{2}y^2]_{y=0}^{y=1} \\
			      &= \frac{2}{5}[2x + \frac{3}{2}] \\
			      &= \frac{4x + 3}{5} \\\\
		\to \mu_X &= \int_{x=0}^{x=1}\frac{4x^2+3x}{5}dx \\
		          &= \frac{\frac{4}{3}x^3 + \frac{3}{2}x^2}{5}|_{x=1} \\\\
		\to \mu_X &= \frac{17}{30}
		\end{align*}
		
		Find $\mu_Y$ by finding h(y):
		
		\begin{align*}
			h(y)  &= \int_{x=0}^{y=1}\frac{2}{5}(2x+3y)dx
		\end{align*}
		
		\subsection{Find the probability that the wait time for the first pump is 3-8 minutes and that the time waiting for pump 2 is 3-7 minutes}
\end{document}

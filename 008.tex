\documentclass[]{article}
\usepackage{amsmath}

%opening
\title{Engineering Statistics Lecture VIII}
\author{Jonathan Bender}
\date{September 19, 2019}

\begin{document}

\maketitle

\begin{abstract}
\end{abstract}

\section{Introduction}
Suppose that we have N equally likely outcomes s.t. $N \geq 1$:
For some event A in the sample set:
P(A) = $\dfrac{1}{n}$

\subsection{Ordered pairs}

Spoz there are $n_1$ ways to do one operation and $n_2$ ways to do a second operation. Let $(x_1, x_2)$ be an ordered pair such that $x_1$ is one of the $n_1$ ways, $x_2$ is one of the $n_2$ ways. So, $n_1$ is the number of manufacturers and $n_2$ is the number of models; (ford, explorer), for instance, is a valid pair.

If operation $x_1$ can be performed $n_1$ ways and likewise with $x_2$ and $n_2$, then there are $n_1 * n_2$ ways to perform both. Therefore, we use the multiplicative rule.

\section{Fast Food Example}
Suppose that we go to Subway to get a footlong sub. We can construct the sandwich using any of the following:
\begin{itemize}
	\item 6 types of bread
	\item 7 meat
	\item 8 cheese
	\item 2 baking styles
	\item 12 types of vegetables
	\item 9 sauces
	\item 4 seasonings
\end{itemize}
So, the number of different types of subs (assuming that we only use one item of each set) is $6 * 7 * 8 * 2 * 12 * 9 * 4 = 290,304$ different ways to construct a footlong sub at Subway.
\section{Lottery Example}
Suppose that you are one of those poor fools that plays the lottery. We want to know what the chances are of us winning. That chance is as follows:
\begin{itemize}
	\item 5 unique numbers called, which range from 1 to 69.
	\item 1 completely independent number ranging from 1 to 26.
\end{itemize}

How many different sets numbers are there to call? Well, let's think about this: each number rules out the possibility to the subsequent numbers existing.
So, we have 69 possibilities for the first number, 68 for the second, 67, 66, 65, and then 26 numbers for the independent number.
By the multiplicative rule, we have $69*68*67*66*65*26$.
This can be rewritten as:
\[\frac{69!}{64!}*26\]

For a fair game, P(win) * the winning total is equal to the cost of playing.

\begin{equation*}
	P(win) * prize(win) = cost()
\end{equation*}

\end{document}